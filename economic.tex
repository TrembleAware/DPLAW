%!TEX root = DPLAW.tex

\setlength{\parskip}{\baselineskip} 
\section{Economic theory}
\begin{frame}[t]
\frametitle{Early Economic Opinion: 1750-1850}
Adam Smith (1776), monopoly was \textit{necessarily hurtful to society}, but a temporary monopoly granted to an inventor was a good way of rewarding his risk and expense.

Jeremy Bentham (1785), comparing rewards by bonus payments with rewards by \textit{exclusive privileges}, held that the latter method was \textit{best proportioned, most natural, and least burdensome} and also \textit{it produces an infinite effect and costs nothing}.

Ludwig Heinrich Jakob (1809) approved of patents only for inventions that had been particularly expensive and \textit{could not just as easily have been made by others}
\end{frame}


\begin{frame}
\frametitle{Early Economic Opinion: 1750-1850}
Simonde de Sismondi (1819), the \textit{dissenter} says that
the result of the privilege granted to an inventor is to give him a monopoly position in the market against the other producers in the country. As a conse quence the consumers benefit very little from the invention, the inventor gains much, the other producers lose, and their workers fall into misery.
\end{frame}

\begin{frame}
\frametitle{The chief argument for patent protection}
\textbf{Ethical grounds} in the name of \textit{justice} or \textit{natural right}

\textbf{Pragmatic grounds} in the name of \textit{promotion of the public interest}
\end{frame}

\begin{frame}
\frametitle{The chief argument for patent protection}
\textit{Natural-law}: man has a natural property right in his own ideas. Appropriation of his ideas by others, that is, their unauthorized use, must be condemned as stealing.

\textit{Reward-by-monopoly}: justice requires that a man receive reward for his services in proportion to their usefulness to society, and that, where needed, society must intervene to secure him such reward.
\end{frame}

\begin{frame}
\frametitle{The chief argument for patent protection}
\textit{Monopoly-profit-incentive}: industrial progress is desirable, that inventions and their industrial exploitation are necessary for such progress, but that inventions and/or their exploitation will not be obtained in sufficient measure if inventors and capitalists can hope only for such profits as the competitive exploitation of all technical knowledge will permit. Society must intervene to increase their profit expectations. Grant temporary monopolies in the form of exclusive patent rights in inventions. 

\textit{Just} reward may serve as an incentive, but often it will not be sufficiently attractive, and either more or something else may be needed to promote technological progress: a bait rather than a just reward.
\end{frame}

\begin{frame}
\frametitle{The chief argument for patent protection}
\textit{Exchange-for-secrets}: presumes a bargain between inventor and society, the former surrendering the possession of secret knowledge in exchange for the protection of a temporary exclusivity in its industrial use. Keep inventions secret. The new technology may only much later become available for general use; indeed, technological secrets may die with their inventors and forever be lost to society.

The patent constitutes a genuine contract between society and inventor; if society grants him a temporary guaranty, he discloses the secret which he could have guarded; quid pro quo, this is the very principle of equity
\end{frame}

\begin{frame}
\frametitle{Discussion of these arguments: Economic Opinion 1850-78}
Natural-law - French Constitutional Assembly - 1791

\textit{every novel idea whose realization or development can become useful to society belongs primarily to him who conceived it, and that it would be a violation of the rights of man in their very essence if an industrial invention were not regarded as the property of its creator}

\textit{complains that something has been stolen which he still possesses, and he wants back something which, if given to him a thousand times, would add nothing to his possession}
\end{frame}

\begin{frame}
\frametitle{Modern Economic Opinion: since 1878}
Ludwig von Mises, speaking of \textit{technological knowledge required for production} as \textit{recipes}, stated:
Such recipes are as a rule free goods as their ability to produce definite effects is unlimited. They can become economic goods only if they are monopolized and their use is restricted.

John R. Commons, the object claimed and owned is merely the expected behavior of other people to be obtained through expected restraint of competition and control of supply.
\end{frame}


\begin{frame}
\frametitle{Modern Economic Opinion: since 1878}
Friedrich von Wieser, monopoly is of limited, duration in order that (ultimately) society may succeed to the unlimited enjoyment of the invention. His invention is the successful outgrowth of a rivalry with others who were experimenting in the same direction as he. Social currents have carried him to his goal. Therefore, after a suitable period of grace, his achievement is once more thrown into the arena of free competition.
\end{frame}


\begin{frame}
\frametitle{Modern Economic Opinion: since 1878}
John Bates Clark, while a patent may sometimes sustain a powerful monopoly it may also afford the best means of breaking one up. Often have small producers, by the use of patented machinery, trenched steadily on the business of great combinations, till they themselves became great producers, secure in the possession of a large field and abundant profit.

Alfred Marshall recognized that \textit{Many giant businesses have owed their first successes to the possession of important patents}
\end{frame}

\begin{frame}
\frametitle{Modern Economic Opinion: since 1878}
Arthur K. Burns, the law with regard to patents rests upon a departure from competition. The prospect of monopoly profits protected by law for a prescribed period is held out as a bait to encourage the improvement of methods of production. The contribution of the patent law to the decline of price competition has passed far beyond the limits suggested by this principle.
\end{frame}

\begin{frame}
\frametitle{Modern Economic Opinion: since 1878}
\textbf{Michael Polanyi}, economist as well as professor of chemistry, writes:
[...] the growth of human knowledge cannot be divided up into such sharply circumscribed phases. Ideas usually develop gradually by shades of emphasis, and even when, from time to time, sparks of discovery flare up and suddenly reveal a new understanding, it usually appears on closer scrutiny that the new idea had been at least partly foreshadowed in previous speculations.
[...] Mental progress interacts at every stage with the whole network of human knowledge and draws at every moment on the most varied and dispersed stimuli.
[...] it is not possible, in general, to attribute to any of them one decisive self-contained mental operation which can be formulated in a definite claim
\end{frame}

\begin{frame}
\frametitle{Modern Economic Opinion: since 1878}
\textbf{Edith T. Penrose}:
One man may spend his life developing a great idea for which society is not ready; another may perfect a bright idea in an evening for a clever gadget which society is willing to buy in large quantities and to pay millions of dollars for. It seems unnecessary to labor the point that there is even less relation between monopoly profits and moral deserts than there is between such profits and the social usefulness of inventions.
\end{frame}

\begin{frame}
\frametitle{Modern Economic Opinion: since 1878}
\textbf{Alfred Marshall} it is generally in the public interest that an improvement \textit{in technology} should be published, even though, it is at the same time patented 

and also

the large manufacturer prefers to keep his improvement to himself and get what benefit he can by using it \textit{without patenting it}
\end{frame}

\begin{frame}
\frametitle{Modern Economic Opinion: since 1878}
\textbf{Friedrich von Wieser} (Austrian theorist) affirms:
The patent right is granted to the inventor, in order to bring his technical leadership, his talents, and genius into the service of society.
\end{frame}

\begin{frame}
\frametitle{Modern Economic Opinion: since 1878}
\textbf{A, T. Hadley}
A patent system, if properly guarded, seems to be thoroughly justified by its results. In the absence of such protection few new inventions would be developed. The risk attending the introduction of a new process is always great. Even when it works thoroughly well in the laboratory or model room, it may not work well in public. The man who first develops a new invention loses his whole capital if it fails. If he is immediately exposed to free competition in case of success, he can enjoy exceptional profits for a short time only. The risk of loss, under such circumstances, outweighs the possibility of gain. [continue]
\end{frame}

\begin{frame}
\frametitle{Modern Economic Opinion: since 1878}
No man will take the lead in a hazardous experiment when those who follow him have practically equal chance of gain and almost no chance of loss. The patent, by making the gain a permanent one, makes it safe for a capitalist to develop a new process. This is the real justification of the system. It has established itself, not primarily as a stimulus for invention or for disclosure, but for utilization and development of new methods requiring the investment of capital and the guaranties which shall make such investment possible.
\end{frame}

\begin{frame}
\frametitle{Modern Economic Opinion: since 1878}
\textbf{Ludwig von Mises}
in the position of an entrepreneur. They have a temporary advantage as against other people. As they start sooner in utilizing their invention themselves or in making it available for use to other people (manufacturers), they have the chance to earn profit in the time interval until everybody can likewise utilize it.
\end{frame}


\begin{frame}
\frametitle{Some basic economic questions}
Are the consumers — the non-patent-owning people — worse off for it?

"No." Patents are granted on inventions which would not have been made in the absence of a patent system; [...] to produce more or better products than could have been produced without them

“Wrong” Many of the inventions for which patents are granted would also be made and put to use without any patent system. The consumers could have the fruits of this technical progress without paying any toll charges.
\end{frame}

\begin{frame}
\frametitle{Some basic economic questions}
Persons with a bent for tinkering and inventing, busy with other jobs during their regular hours, may be glad to use their free evenings and weekends for inventive activity. Others, employed in research and development, may be willing to work overtime. This second pool of potential resources may be of great importance for the implementation of \textit{crash programs} of research and development in a national emergency.
\end{frame}

\begin{frame}
\frametitle{Some basic economic questions}
The available productive resources are allocated among four uses:
\begin{itemize}
	\item the production of consumers goods
    \item the production of capital goods
    \item the production of knowledge
    \item the production of security from invasion and revolution
\end{itemize}
\end{frame}

\begin{frame}
\frametitle{Some basic economic questions}
The production of knowledge may likewise be so divided, because trained people who retire or die must be replaced by young persons who have to be trained and educated, so that the maintenance of an existing stock of knowledge re quires constant replacement, and only a part of the resources devoted to the production of knowledge can, through research and development, increase the stock of existing knowledge.
\end{frame}

\begin{frame}
\frametitle{Some basic economic questions}
Consumption can be increased if the accumulation of capital and knowledge is increased. But, alas, such accumulation presupposes the availability of resources, and from where can they come? If resources have been fully used, increased appropriations for investment in capital and knowledge imply reduced appropriations to the production of consumers goods. There is, therefore, a dilemma: The way to increased consumption is first to reduce it.
\end{frame}

\begin{frame}
\frametitle{Some basic economic questions}
\textbf{Increased research and development in order to increase the stock of knowledge is a splendid thing for society}

\textbf{A choice by society to increase research and teaching implies a choice, though usually unconscious, to have in the next years less productive equipment or less consumption, or less of both, than they might have had. Should a relative cut-back of consumption prove impracticable, the choice is between \textit{knowledge} and \textit{equipment}.}
\end{frame}

\begin{frame}
\frametitle{Competitive research, waste, and serendipity}
Different firms and different research teams competing with one another in finding solutions to the same research problem in the same field.
\begin{enumerate}
	\item to be the first to find a patentable solution to a problem posed by the needs and preferences of the customers
    \item to find an alternative solution to the same problem in order to be able to compete with him in the same market
    \item \textit{block} competitor's efforts to \textit{invent around} the first patent
\end{enumerate}
\end{frame}


\begin{frame}
\frametitle{Competitive research, waste, and serendipity}
\textit{Justification} for this kind of \textit{competitive research}: it can be summarized in the colorful word \textit{\textbf{serendipity}}. 

\textit{Serendipity}: the faculty of making happy and unexpected discoveries by accident
\end{frame}

\begin{frame}
\frametitle{Some confusions, inconsistencies, and fallacies}
Patent protection is exchanged for the disclosure of secrets.

Great benefits are obtained for society by securing the general availability.

Notion of the inventor’s \textit{natural property right} in the invention not to be confused with the property right in the patent
\end{frame}


\begin{frame}
\frametitle{Some confusions, inconsistencies, and fallacies}
Value of patents
\begin{itemize}
	\item the value of patents to their owners
    \item the value of patents to society
    \item the value of the patent system to society
    \item the value of patented inventions to their users
    \item the value of patented inventions to society
    \item the value of patentinduced inventions to society.   \end{itemize}
\end{frame}

\begin{frame}
\frametitle{Some confusions, inconsistencies, and fallacies}
polio vaccine, Dr. Salk, generously contributed his idea to society without applying for a patent.
\end{frame}


\begin{frame}
\frametitle{Private and social cost and value: explaining basic economic concepts}
\begin{itemize}
	\item private cost: are the money expenses which a producer has to incur in the production of his output
    \item social cost: 
    \item private value: producer's total of money receipts from the sale of his output
    \item social value: Often, society, or some members of society, will find that they can enjoy an incidental advantage for which nothing is paid to the producer
\end{itemize}
Private cost and social cost will differ when the producer’s money expenses do not reflect the displeasures or sacrifices caused to others.
\end{frame}


\begin{frame}
\frametitle{The cost and value of inventions}
\textit{Social value of inventions}

The social value of a particular invention; the social value of the annual crop of invention patented or unpatented; the social value of the annual crop of patented inventions; and, lastly, the social value of the annual crop of patent-generated inventions, that is, of inventions that would not have been made or developed had it not been for the incentives afforded by the patent system.
\end{frame}

\begin{frame}
\frametitle{The cost and value of additional inventions}
\textit{May we “dream up” some experimental testing of the differences between a world with patents and one without patents?}

1700 and 1750 might show superior progress in the specimen equipped with patent systems; the worlds of 1800 might show no differences in the rates of progress; and the worlds of more recent vintage might show faster progress in the specimen without patents.

\textit{increment of invention} (attributable to the patent system) be cause inventions can be neither counted nor weighed nor measured in any practical way. Inventions can often be sub divided or fused, and hence counting is arbitrary.
\end{frame}


\begin{frame}
\frametitle{The cost and value of additional inventions}
\begin{itemize}
	\item the operating cost of the patent system
    \item the cost of inventing
    \item the cost of innovating
    \item the cost of immanent restrictions in the use of patented inventions
    \item the cost of transcendent restrictions upon production as a result of general monopoly control strengthened through patent positions
    \item the cost of obstructions and encumbrances to potential inventors and innovators
\end{itemize}
\end{frame}



\begin{frame}
\frametitle{Shortening or lengthening the duration of patents}
\begin{itemize}
	\item the increase in the duration of patents by 1 year may increase profits
    \item the opportunity cost a of using them for other purposes, induce an increase in current expenditures for research and development
    \item increase in the demand for physicists, chemists, engineers, and all sorts of specialists; and may, depending on the supply of such human re sources, lead to a transfer of manpower from various activities and thus to an increase in manpower allocated to research and development
    \item increase number of new and useful technological ideas
\end{itemize}
\end{frame}

\begin{frame}
\frametitle{Shortening or lengthening the duration of patents}
\begin{itemize}
    \item include an enlarged portion of duplicate or substitute inventions, or of otherwise unusable inventions
    \item increase of the output per unit of productive services
    \item an increase of the national product
\end{itemize}
\end{frame}

\begin{frame}
\frametitle{Introducing or abolishing compulsory licensing}
Compulsory licensing would probably reduce the incentive effects of a patent system, but increase the rate of utilization of the patented techniques that have proven themselves commercially successful.

Considerations of the comparative effectiveness of the system in stimulating invention and of the comparative rates of utilization of patented technology that has proven itself commercially successful.
\end{frame}