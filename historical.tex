%!TEX root = DPLAW.tex

\setlength{\parskip}{\baselineskip} 
\section{Historical Survey}
\begin{frame}[t]
\frametitle{Early History (Before 1624)}

Oldest example: $14^{th}$ century. Probably the first “patent law”, was enacted in 1474 by the Republic of Venice.

In the $16^{th}$ century, patents were widely used by \textit{German} princes, some of whom had a well-reasoned policy of \textit{granting privileges} on the basis of a careful consideration of the utility and novelty of the inventions and, also, of the burden which would be imposed on the country by \textit{excluding} others from the use of these inventions and by enabling the patentees to \textit{charge higher prices}.

\end{frame}

\begin{frame}
\frametitle{Early History (Before 1624)}
Exclusive privileges were on:
\begin{itemize}
	\item new inventions
	\item skilled crafts imported from abroad
\end{itemize}

Canton Bern in Switzerland granted in 1577 to inventor Zobell a \textit{permanent exclusive privilege}
\end{frame}

\begin{frame}
\frametitle{Early History (Before 1624)}
In France, the persecution of innovators by guilds of craftsmen continued far into the
18th century. 

In 1726, the weavers' guild threatened design printers with severe punishment, including death.

Royal patent privileges were sometimes conferred, not to grant exclusive rights, but to grant permission to do what was prohibited under existing rules.
\end{frame}

\begin{frame}
\frametitle{The Spread of the patent system (1624-1850)}
The Statute of Monopolies is the basis of the present British patent law, and became the model for the laws elsewhere.
\begin{itemize}
	\item 1691 South Carolina - first \textit{general} patent law
	\item 1762 France - edict of King Louis XV did little more than prohibit permanent privileges and provide for inventors’ patents limited to 15 years
    \item 1791 Constitutional Assembly passed a comprehensive patent law, in which the inventor's right in his creation was declared a \textit{property right} based on the \textit{rights of man}
\end{itemize}
\end{frame}

\begin{frame}
\frametitle{The Spread of the patent system (1624-1850)}
\begin{itemize}
	\item 1787 United States of America - Constitution had given Congress the power \textit{to promote the Progress of Science and useful Arts, by securing for limited Times to Authors and Inventors the exclusive Right to their respective Writings and Discoveries}
	\item 1794 Austria (Hofdekret) - announced the establishment of a patent system
\end{itemize}
\end{frame}

\begin{frame}
\frametitle{The rise of an antipatent movement (1850- 1873)}
\begin{itemize}
	\item 1863 Germany - condemned \textit{patents of invention as injurious to common welfare}
	\item 1849, 1851, 1854, and twice in 1863 Switzerland - \textit{economists of greatest competence} had declared the principle of patent protection to be \textit{pernicious and indefensible}
\end{itemize}
\end{frame}

\begin{frame}
\frametitle{The victory of the patent advocates (1873-1910)}
Thanks to the bad crisis, public opinion had turned away from \textit{the pernicious theory of free competition and free trade}
\begin{itemize}
	\item 1874 Britain - drastic reform bill had passed the House of Lords withdrew in the House of Commons
	\item 1877 Germany - uniform patent law for the entire Reich
    \item 1872 Japan - had adopted her first patent law, only to abolish it again in 1873, enacted another law in 1885
\end{itemize}
\end{frame}


