%!TEX root = DPLAW.tex

\setlength{\parskip}{\baselineskip} 
\section{Institutional Facts and Problems}
\begin{frame}[t]
\frametitle{Conditions, procedures, and limits of patent protection}

A patent confers the right to secure the enforcement power of the state in excluding unauthorized persons from making commercial use of a clearly identified, novel, and useful invention.

An invention is a new contrivance, device, or technical art newly created, in contrast to a discovery of a principle or law of nature that has already \textit{existed} though unknown to man.

\textit{invention}: 
\begin{itemize}
	\item it must be an \textit{unusual mental achievement}
	\item \textit{flash of genius}$?$
\end{itemize}
\end{frame}

\begin{frame}
\frametitle{Conditions, procedures, and limits of patent protection}
Patentability shall not be negatived by the manner in which the invention was made.

\textit{Subjective novelty} is universally rejected in favor of objective tests, such, as \textit{not previously patented, published or used}, is understood; but whether the reinvention of a forgotten art or the introduction or importation of a foreign art should be patentable are controversial questions, depending on the purposes patent protection is supposed to serve.

Industrial and commercial application of the invention
\end{frame}


\begin{frame}
\frametitle{Conditions, procedures, and limits of patent protection}
\textit{Who is to judge the patentability of an invention and at what stage of the game, have received different answers, and different procedures have been adopted in different countries}$?$

Under the registration system the validity of a registered patent is examined only if an interested party attacks it in the courts and asks that the patent be invalidated
\end{frame}

\begin{frame}
\frametitle{Conditions, procedures, and limits of patent protection}
\textit{Interference proceedings}: when the Office finds that two or more pending applications seem to claim, partly or wholly, the same invention.

\textit{examination-plus-opposition system}: provides for an interval of time after publication of the specifications examined and accepted by the official examiner and before the issuance of the patent, in order to enable interested persons to oppose the patent grant.
\end{frame}

\begin{frame}
\frametitle{Conditions, procedures, and limits of patent protection}
The registration system is administratively the \textit{cheapest} but may \textit{burden} the economy with the cost of exclusive rights being exercised for many inventions which, upon examination, would have been found \textit{nonpatentable}. In favor of the examination system it has been said that it \textit{avoids a mass of worthless, conflicting, and probably invalid patents}, onerous to the public as well as to bona fide owners of valid patents; that it prevents the fraudulent practice of registering and selling patents similar to the claims being patented by others; and that it drastically reduces the extent of court litigation. The latter advantage, however, may not be realized if Patent Office and courts apply different standards of patentability.
\end{frame}

\begin{frame}
\frametitle{Conditions, procedures, and limits of patent protection}
Germany (and until 1949 in England) no patents could be granted for inventions of new food products or new medicines.

\textit{If patents are regarded as means of stimulating technological progress, and if progress in the food and drug industries is not less desired than in other industries, why should these exceptions be made?}

If society aims at \textit{stimulating} innovation and at \textit{attracting} venture capital into pioneering investment, then the controversies about the nature of \textit{inventions} are beside the point. After all, the innovators’ risks are \textit{not} proportional to the costs and results of the inventive efforts.
\end{frame}


\begin{frame}
\frametitle{Conditions, procedures, and limits of patent protection}
Duration of patents - English patents

14-year term after 1624 was based on the idea that 2 sets of apprentices should, in 7 years each, be trained in the new techniques, though a prolongation by another 7 years was to be allowed in exceptional cases.
\end{frame}

\begin{frame}
\frametitle{Conditions, procedures, and limits of patent protection}
Period of inventors' protection (proposal):
\begin{itemize}
	\item for the rest of his life
    \item for the average length of time for which a user of an invention might succeed in keeping it secret
    \item average time it would take for others to come up with the same invention
    \item average period in which investments of this kind can be amortized
    \item eternal protection through perpetual patents
\end{itemize}
\end{frame}


\begin{frame}
\frametitle{Conditions, procedures, and limits of patent protection}
\textit{Period of inventors' protection (actual fact)}:

Patent terms were lengthened to 15, 16, 17 and 18 years in most countries, and to 20 years in some.
\end{frame}


\begin{frame}
\frametitle{\textit{Abuse} of the patent monopoly}
Abuse of the patent monopoly happened when the social objectives which it is supposed to serve are not promoted but rather jeopardized by the way it is used.\\
\textit{Extending the time period of control}:
\begin{itemize}
	\item delays in the pendency of the patent between application and issuance
	\item secret use of the invention prior to the application for a patent or incomplete disclosure
    \item successive patenting of strategic improvements of the invention which make the unimproved invention commercially unusable after expiration of the original patent
    \end{itemize}
\end{frame}

\begin{frame}
\frametitle{\textit{Abuse} of the patent monopoly}
\begin{itemize}
    \item creation of a monopolistic market position based on the goodwill of a trademark associated with the patented product or process, where the mark and the consumer loyalty continue after expiration of the patent
    \item licensing agreements which survive the original patent because they license a series of existing improvement patents and a possibly endless succession of future patents.
\end{itemize}
\end{frame}

\begin{frame}
\frametitle{\textit{Abuse} of the patent monopoly}
\textit{Extending the scope and strength}:
\begin{itemize}
	\item basic patent: on a bona fide basic invention
    \item umbrella patent: where illegitimately broad or ambiguous claims, covering the entire industry
    \item bottleneck patent: which is not basic but good enough to hold up or close the entire industry, by an \textit{aggregation} or \textit{accumulation} of patents which secure domination of all existing firms and effectively close the industry to newcomers, or by the use of restrictive licensing
agreements establishing domination or cartelization of the industry and exclusion of newcomers.
\end{itemize}
\end{frame}

\begin{frame}
\frametitle{\textit{Abuse} of the patent monopoly}
In some countries, especially in England, \textit{insufficient working} is regarded as an abuse of the patent monopoly, as is also the charging of excessive prices for patented articles.

Abuses are merely some of the social costs inherent in the patent system and are only rarely connected with any malpractices on the part of patentees.
\end{frame}

\begin{frame}
\frametitle{Compulsory licensing}
Among the sanctions provided by various patent laws for \textit{abuses} of patent protection are revocation of patents, refusal of judicial relief in infringement suits, and compulsory licensing.

In Germany the most frequent reason has been the existence of dependent patents, that is, of patents covering inventions which could not be worked without license under a patent held by someone else.

In England insufficient use of a patent may in the future become a more frequent reason for compulsory licensing or for \textit{licenses of right}
\end{frame}


\begin{frame}
\frametitle{Compulsory licensing}
Patentees could hope for such revenues as they would collect as royalties from their licensees and as \textit{differential rents} due to the cost advantage over their royalty-paying competitors.\\
These revenues might not be smaller than the potential monopoly profits in cases of relatively less strategic inventions, but they would probably be much smaller in cases of basic inventions and in all other instances where a strong patent position could permit a firm to control some of its markets.
\end{frame}

\begin{frame}
\frametitle{Plans for reforms and alternatives to the patent system}
Recommendations:
\begin{itemize}
	\item maintain the highest standard of invention
    \item avoid broad claims
    \item insist on more adequate disclosure
    \item publicize patent applications and establish opposition procedures
    \item improve examination procedures
    \item apply \textit{economic as well as technological tests in determining whether to grant the patent}
    \item abandon the flash-of-genius notion in favor of explicit consideration of the size of research expenditures required for inventive and developmental activity
    \item reasonable use of the invention
\end{itemize}
\end{frame}


\begin{frame}
\frametitle{Plans for reforms and alternatives to the patent system}
Proposal:\\
\begin{itemize}
	\item In order to combine the advantages of \textit{free accessibility of inventions to all}, insured through general licenses of right, with the benefits of adequate incentives to investors in research and innovation, a proposal was to supplement licenses of right by government regards to patentees on a level ample enough to give general satisfaction to inventors and their financial promoters.
	\item 1787 United States - discussions of the powers to be reserved for Federal legislation, Madison proposed a premium system instead of a patent system
\end{itemize}
\end{frame}

\begin{frame}
\frametitle{Plans for reforms and alternatives to the patent system}
\begin{itemize}
	\item 1834 Russia - established a commission to determine awards for inventors in lieu of exclusive privileges
\end{itemize}
The United States has not only maintained a very strong patent system but has also resorted to subsidized research and to Government research.\\
The greater part of the total research expenditures in the United States is now financed by the Government. In 1953 the Federal Government contributed \$2.8 billion or 52 percent of the total funds spent on research and development.
\end{frame}

\begin{frame}
\frametitle{International Patent Relations}
The existence of national patent systems, in a world with expanding international trade, raised problems which soon suggested the desirability of international understandings.\\
Advocates of industrialization were interested in domestic production and, therefore, opposed to a system that would protect the importer from the domestic producer, instead of the producer from the importer. Internationalists found it preposterous that a patentee should be forced to forego the cost advantages of large-scale production and to manufacture in 20 or more different countries with compulsory-working provisions.
\end{frame}

\begin{frame}
\frametitle{International Patent Relations}
The oldest international agreements involving patent - German - second quarter of the $19^{th}$.\\
First multilateral agreement - German Zollverein - 1842.\\
First International Patent Congress was held in 1873 in Vienna, the next two in 1878 and in 1880 in Paris; in 1884 the International Union for the Protection of Industrial Property was created, with a permanent secretariat, the International Bureau for the Protection of Industrial Property, in Bern, Switzerland.
\end{frame}

\begin{frame}
\frametitle{International Patent Relations}
International Patent:
\begin{itemize}
	\item foreigners (nationals of Union countries) shall receive in each country the same treatment as the nationals of that country
    \item an applicant for a patent on an invention in one country shall be given the advantage of that date of application in other Union countries provided application is made in the latter within 12 months of the original application (the so-called priority clause)
    \item patents in each country shall be independent of patents on the same invention in other countries—particularly they shall not be affected by refusal, revocation, or expiration in any other country
\end{itemize}
\end{frame}

\begin{frame}
\frametitle{International Patent Relations}
\begin{itemize}
    \item importation by the patentee of goods produced in other Union countries shall not entail forfeiture of patent protection for these goods
    \item each country may take measures to prevent abuses resulting from the exclusive rights conferred by patents, such as \textit{failure to use}, but it may revoke these patents only if compulsory licensing should be an insufficient remedy—and compulsory licenses cannot be required until 3 years after issuance of a patent and only if the patentee does not produce acceptable excuses.
\end{itemize}
\end{frame}

\begin{frame}
\frametitle{International Patent Relations}
Countries with strong patent positions have often prodded and put pressure on weaker countries to adopt patent systems.
\end{frame}
